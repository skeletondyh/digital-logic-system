\section{编译仿真与前端仿真}

本部分主要说明本组项目中后端modelsim编译仿真与前端实时仿真模块实现的细节。


编译仿真:

我们通过脚本调用modelsim进行vhdl文件的编译仿真。脚本文件位于public/files中,文件名为model.sh,相关的modelsim编译与仿真命令如下:

\begin{table}[h!]
\centering  
\caption{modelsim编译仿真命令}  
\begin{tabular}  
{>{\columncolor{white}}rc}  
\toprule[1pt]  
\rowcolor[gray]{0.9} 命令    &描述\\  
\midrule  
{filename=param1 \\ jiliname=param2 }   &\multicolumn{1}{>{\columncolor{white}[0pt][0pt]}c}{vhdl源文件名和激励文件文件名分别 为参数1和参数2 } \\

vlib -unix work &\multicolumn{1}{>{\columncolor{white}[0pt][0pt]}c}{创建工作库work}   \\  
vcom filename.vhd jiliname.vhd   &\multicolumn{1}{>{\columncolor{white}[0pt][0pt]}c}{编译源文件和激励文件 }    \\ 
vsim -c -do "vcd add wave /\$\{jiliname\}/*" -do "run 3000ns" -do "quit -sim" -do "q" \$\{jiliname\}   &\multicolumn{1}{>{\columncolor{white}[0pt][0pt]}c}{通过激励文件对源文件进行仿真,仿真时间3000ns} 
\bottomrule[1pt]  
\end{tabular}  
\end{table}  

vcd文件解析

modelsim仿真会生成.wlf文件,通过脚本文件可以将.wlf文件转化为通用 的.vcd文件。为了了生成波形图形所需的数据,我们需要对.vcd文件进行解 析。我们使用node.js中的Duplex流对文件进行了处理。主要函数列表如下: 

\begin{table}[h!]
\centering  
\caption{modelsim编译仿真命令}  
\begin{tabular}  
{>{\columncolor{white}}rcc}  
\toprule[1pt]  
\rowcolor[gray]{0.9} 函数名称    &说明 &返回值\\  
\midrule  
state (initial, props)   &\multicolumn{1}{>{\columncolor{white}[0pt][0pt]}c}{将.vcd 文件分为不同的部 分,将文件的每一行标注为不同的状态(是否为波形数据、是否是仿真时间)}  &{该行文件的状态}\\

vcdStream (opts)  &\multicolumn{1}{>{\columncolor{white}[0pt][0pt]}c}{dshengcheng文件流}  &文件流 \\  
\bottomrule[1pt]  
\end{tabular}  
\end{table}  


前端仿真:

在本项目中,我们基于陈雅正小组的数电平台一期项目,进一步实现了开关、50Hz时钟、灯泡、74系列芯片、七段数码管、数字译码器、Vcc、GND的完整功能的仿真。前端仿真基于Draw2D库实现,其基本的实现思想是,对于组合电路,使用一个全局的计数器作为时钟,每个时钟周期电平信号都会从每根导线连接的输入端传播到输出端,进而实现信号的传播,而对于时序电路,使用局部变量来模拟其记忆功能。

基本架构:
前端的仿真的实现思想为继承draw2d库中提供的基类与模板类,通过重载类方法来加入新的功能。基本框架的代码位于View.js中。



\begin{table}[h!]
\centering  
\caption{AnotherLabelInPlaceEditor}  
\begin{tabular}  
{>{\columncolor{white}}rc}  
\toprule[1pt]  
\rowcolor[gray]{0.9} 方法    &功能描述\\  
\midrule  
init(listener, view) &\multicolumn{1}{>{\columncolor{white}[0pt][0pt]}c}{调用基类的构造函数,初始化监听器和界面引用} \\
commit &\multicolumn{1}{>{\columncolor{white}[0pt][0pt]}c}{处理提交请求,检查命名的合法性}   \\  

\bottomrule[1pt]  
\end{tabular}  
\end{table}  


\begin{table}[h!]
\centering  
\caption{OrthogonalConnectionCreatePolicy}  
\begin{tabular}  
{>{\columncolor{white}}rc}  
\toprule[1pt]  
\rowcolor[gray]{0.9} 方法    &功能描述\\  
\midrule  
onClick(figure, x, y, shiftKey, ctrlKey) &\multicolumn{1}{>{\columncolor{white}[0pt][0pt]}c}{figue为对应的图形对象,x、y坐标,shiftKey, ctrlKey为对应t功能键是否按下,本函数的作用为监听鼠标与键盘的图形操作事件并作出响应} \\

\bottomrule[1pt]  
\end{tabular}  
\end{table}  

\begin{table}[h!]
\centering  
\caption{tot.View}  
\begin{tabular}  
{>{\columncolor{white}}rc}  
\toprule[1pt]  
\rowcolor[gray]{0.9} 方法    &功能描述\\  
\midrule  
init(app, id, limits) &\multicolumn{1}{>{\columncolor{white}[0pt][0pt]}c}{app为当前所在的项目, id为对象的标识符,limits为对象的数量限制,初始化view对象} \\

setZoom(newZoom) &\multicolumn{1}{>{\columncolor{white}[0pt][0pt]}c}{newZoom为zoom对象,控制当前画布的缩放} \\

getBoundingBox() &\multicolumn{1}{>{\columncolor{white}[0pt][0pt]}c}{获取当前画布中所有元件的包围盒} \\

onDrag() &\multicolumn{1}{>{\columncolor{white}[0pt][0pt]}c}{监听拖拽事件} \\

onDrop() &\multicolumn{1}{>{\columncolor{white}[0pt][0pt]}c}{监听释放事件} \\

checkLimit() &\multicolumn{1}{>{\columncolor{white}[0pt][0pt]}c}{检查元件数量限制} \\

initRemain()/updateRemain()/getRemain() &\multicolumn{1}{>{\columncolor{white}[0pt][0pt]}c}{元件剩余数量的操作} \\

simulationStart() &\multicolumn{1}{>{\columncolor{white}[0pt][0pt]}c}{开始仿真输入电路} \\

\_calculate &\multicolumn{1}{>{\columncolor{white}[0pt][0pt]}c}{原件内部逻辑的计算,需被具体的实例重载} \\

\bottomrule[1pt]  
\end{tabular}  
\end{table}  

74系列芯片的前端实现
以74LS00为例,对74系列芯片的前端实现作说明:74CLS00继承draw2d.SetFigure类,定义一个图形对象,在init()定义其端口、形状、最大伞入伞出数等属性,同时重载calculate()方法,定义电平信号在该芯片内传播的逻辑,进而实现芯片的逻辑功能。

\begin{table}[h!]
\centering  
\caption{74LS00}  
\begin{tabular}  
{>{\columncolor{white}}rc}  
\toprule[1pt]  
\rowcolor[gray]{0.9} 方法    &功能描述\\  
\midrule  
init(attr, setter, getter)&\multicolumn{1}{>{\columncolor{white}[0pt][0pt]}c}{调用父类的构造函数,获取attr、setter、getter,同时定义芯片的形状、接口等属性} \\

createShapeElement()&\multicolumn{1}{>{\columncolor{white}[0pt][0pt]}c}{创建芯片的图形} \\

createSet()&\multicolumn{1}{>{\columncolor{white}[0pt][0pt]}c}{创建芯片的Label等信息} \\

getPersistentAttributes()&\multicolumn{1}{>{\columncolor{white}[0pt][0pt]}c}{获取芯片对象的不变属性} \\

setPersistentAttributes()&\multicolumn{1}{>{\columncolor{white}[0pt][0pt]}c}{设置芯片属性} \\

onStart()&\multicolumn{1}{>{\columncolor{white}[0pt][0pt]}c}{监听开始仿真事件} \\

onStop()&\multicolumn{1}{>{\columncolor{white}[0pt][0pt]}c}{监听停止仿真事件} \\

calculate()&\multicolumn{1}{>{\columncolor{white}[0pt][0pt]}c}{重载计算函数,定义芯片内部逻辑} \\
\bottomrule[1pt]  
\end{tabular}  
\end{table}  


在portMap.js中,我们实现了前端输入的电路数据到后端输入给VHDL转换模块的电路数据的格式的转换与统一,解决了数据格式的不一致性问题。另外,在portMap中我们还加入了电路连接的合法性检查函数,由于Draw2D框架及之前项目的实现存在一些漏洞,无法直接判断连接的两端是否与端口属性匹配,故需进行额外的逻辑判断,若有连接错误则进行修复。


\begin{table}[h!]
\centering  
\caption{portMap变量与方法说明}  
\begin{tabular}  
{>{\columncolor{white}}rc}  
\toprule[1pt]  
\rowcolor[gray]{0.9} 变量/方法    &功能描述\\  
\midrule  
chip_map &\multicolumn{1}{>{\columncolor{white}[0pt][0pt]}c}{前端芯片名称到后端芯片类型的映射的字典} \\
port_map &\multicolumn{1}{>{\columncolor{white}[0pt][0pt]}c}{每一种前端芯片的端口命名,到后端端口种类与名称的映射的字}   \\  
port_Map(json): &\multicolumn{1}{>{\columncolor{white}[0pt][0pt]}c}{json为前端传入的json格式的电路数据,本函数的作用是将前端传入的电路数据转换成与VHDL转换模块相匹配的格式的后端电路数据,并且调用connection_Correct(),检查数据合法性。}    \\ 
connection_Correct(json): &\multicolumn{1}{>{\columncolor{white}[0pt][0pt]}c}{json为前端传入的json格式的电路数据,本函数的作用是检查每一个connection对象是否合法,即connection的source与destination是否与端口的属性相匹配,若不匹配则报错或进行更正。} 
\bottomrule[1pt]  
\end{tabular}  
\end{table}  

\iffalse
chip_map : 前端芯片名称到后端芯片类型的映射的字典
port_map : 每一种前端芯片的端口命名,到后端端口种类与名称的映射的字典

function port_Map(json):
json为前端传入的json格式的电路数据,本函数的作用是将前端传入的电路数据转换成与VHDL转换模块相匹配的格式的后端电路数据,并且调用connection_Correct(),检查数据合法性。

function connection_Correct(json) : 
json为前端传入的json格式的电路数据,本函数的作用是检查每一个connection对象是否合法,即connection的source与destination是否与端口的属性相匹配,若不匹配则报错或进行更正。
\fi

\end{document}
