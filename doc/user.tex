\documentclass{article}
\usepackage{CJKutf8}
\usepackage{indentfirst}
\begin{document}
\begin{CJK}{UTF8}{gbsn}
\CJKindent
\section{用户端功能}

\par用户端功能主要分为电路拖拽编辑、VHDL代码编辑、项目信息显示、提交作业四大部分。本工程在第一期工程的基础上重构了整个电路拖拽编辑的架构、添加了提交作业的功能,并在仿真与VHDL代码编辑方面修改了若干bug,增加了部分新特性,使用户体验更佳。

\subsection{电路拖拽编辑}
    \par本项目中使用了更为成熟的Draw2D框架来代替一期工程中的Konva.js,并在原本在线拖拽搭建电路的基础上,添加了实时仿真、撤销操作等功能。
    \par下面介绍各个部件的实现细节:
\subsubsection{背景画布}
    \par背景画布以draw2D.Canvas为基类,使用了draw2D库中自带的多个policy(也即特性),并进行了修改以满足电路编辑需求。
    \par使用的policy列表如下:
    \begin{table}[!h]
    \begin{tabular}{|l|l|}
    \hline
    policy名字 & policy作用 \\
    \hline
    DropInterceptorPolicy & 接受Droppable对象 \\
    \hline
    DragConnectionCreatePolicy & 允许通过拖拽连接点产生连线 \\
    \hline
    OrthogonalConnectionCreatePolicy & 允许通过点击连接点产生连线 \\
    \hline
    CoronaDecorationPolicy & 鼠标接近时显示并高亮连接点,否则隐藏 \\
    \hline
    ShowGridEditPolicy & 在背景上显示网格,方便编辑与观察 \\
    \hline
    SnapToGeometryEditPolicy & 自动对齐部件 \\
    \hline
    SnapToCenterEditPolicy & 自动对齐部件 \\
    \hline
    SnapToInBetweenEditPolicy & 自动对齐部件 \\
    \hline
    EditEditPolicy & 编辑模式下使用,允许用户修改电路图 \\
    \hline
    SimulationEditPolicy & 实时仿真模式下使用,只允许用户查看电路图 \\
    \hline

    \end{tabular}
    \caption{policy列表}
    \end{table}

    \par为了方便用户使用,我们还使用了mousetrap库来捕获用户的按键输入,以提供快捷键功能。目前支持用户使用上下左右
    键移动画布与使用delete键删除元件。

\subsubsection{拖拽与缩放}
    \par用户可以通过shift+鼠标拖拽的方式来移动画布。上文中提到的两个EditPolicy中都添加了对shift按键的处理,在捕获到事件后,画布会随着鼠标的移动调整自身与网页的相对位置。
    \par同时,用户可以依靠画布右下角的缩放按钮来调整画布大小。由于Draw2d框架中自带setZoom方法用来调整背景画布的缩放与getZoom方法用来获取当前缩放,程序中只需要分别调用这些接口即可。
    \begin{table}[!h]
    \begin{tabular}{|l|l|l|}
    \hline
    按键 & 效果 & 代码段 \\
    \hline
    + & 放大画布及元件 & setZoom(this.getZoom()*1.2) \\
    \hline
    - & 缩小画布及元件 & setZoom(this.getZoom()*0.8) \\
    \hline
    100\% & 恢复原缩放 & setZoom(1.0) \\
    \hline
    \end{tabular}
    \caption{缩放相关函数}
    \end{table}

\subsubsection{元件}
    \par此次工程中的元件包含数字电路实验中的所有实验芯片及Vcc、GND、七段数码管、2-10进制转换器。所有芯片继承自draw2d.SetFigure,输入端使用DecoratedInputPort,输出端使用普通Port,标签为"output"。输入输出端只能接另一种端口,同种端口内部互联时网页会自动阻止操作。
    \par每个元件包含以下自定义属性及方法:
    \begin{table}[!h]
    \begin{tabular}{|l|l|}
    \hline
    名称 & 描述 \\
    \hline
    name & 芯片名称 \\
    \hline
    note & 芯片功能描述 \\
    \hline
    tagForWave & 元件标签,由用户搭建电路时编辑 \\
    \hline
    calculate & 实时仿真时每一个周期计算步骤\\
    \hline
    \end{tabular}
    \caption{元件参数及接口}
    \end{table}

    \par一个元件示例:
    \begin{table}[!h]
    \begin{tabular}{|l|l|}
    \hline
    名称 & 内容 \\
    \hline
    name & 74LS00 \\
    \hline
    note & "Y = not (A and B)" \\
    \hline
    calculate & Y1.setValue(!(A1.getValue() \&\& B1.getValue())); (其他端口同理) \\
    \hline
    \end{tabular}
    \caption{74LS00相关参数}
    \end{table}

    \par实际生成代码时,开关、50Hz脉冲等元件会被视为输入端口,小灯泡、七段数码管会被视为输出端口。
    \par元件列表在网页上会以画布外的悬浮列表形式显示,分为输入、输出与芯片三个部分,点击相应类别标题可以收起或弹出元件列表。当点击某一具体元件时,会生成一个带有元件类别信息的draw2d\_droppable对象,拖拽该对象进画布会触发画布的onDrop事件,进而在画布对应位置生成元件。
    \par每个元件的剩余数量会在元件列表中显示,onDrop事件会检查并修改剩余数量,并更新网页;删除元件时,同样会修改剩余数量,更新网页。

\subsubsection{标签}
    \par为了生成与电路图对应的VHDL代码用来仿真,每一个输入输出端口需要有唯一的标识符,也即标签。程序中通过jQuery的右键菜单功能允许用户右键点击元件来添加标签,或右击标签来删除标签。
    \par标签功能通过修改draw2d中的LabelInplaceEditor,自定义类AnotherLabelInPlaceEditor实现,支持修改和拖拽标签。
    \par接口:
    \begin{table}[!h]
    \begin{tabular}{|l|l|}
    \hline
    名称 & 内容 \\
    \hline
    commit & 修改完毕后提交修改内容 \\
    \hline
    cancel & 放弃修改内容 \\
    \hline
    \end{tabular}
    \caption{标签接口}
    \end{table}
    \par整个程序会维护一个已使用的名字集合,创建输入输出元件时默认命名为inx,outx(x为该类型元件总数量)。当用户试图修改标签或为其他元件添加新标签时,程序会检查以下三点合法性:
    \par1.通过正则表达式检查该名字是否符合VHDL命名规范,是否过长或为空;
    \par2.通过正则表达式检查该名字是否为VHDL保留字;
    \par3.检查是否与其他元件重名。
    \par如果不符合要求,程序会弹出对话框显示对应错误信息并取消修改,否则接受修改。此外,输入输出元件上的标签有特殊的标记,删除这些标签时网页会发出提示并拒绝修改,其他标签则可正常删除。
    \par每个标签上都安装了一个listener,会在标签的commit事件触发后修改相连接元件的tagForWave属性,供生成代码模块确定其名字。
    
\subsubsection{撤销与重做}
    \par二期工程中添加了撤销与重做的功能,当用户对之前做的操作不满意时,可以点击工具栏上的按钮来进行撤销与重做操作。
    \par Draw2d框架中自带了CommandStack功能,可以以栈的方式记录物体的添加、删除与移动操作,这部分解决了撤销的需求,但关联的属性仅限于画布基类内部,无法将元件剩余数量、元件名字同步更改。为此,我们改写了commandStack的stackChanged事件,在事件内部另外处理这些变更。具体处理的变更有:
    \par1.撤销元件的删除/添加操作时,修改元件剩余数量;
    \par2.撤销对标签的编辑时,修改相应元件的tagForWave属性;
    \par3.撤销生成/删除标签时,修改已用过的标签名集合。
    \par4.撤销栈或重做栈为空时,将对应按键设为灰色不可用状态。

\subsubsection{实时仿真}
    \par本工程支持实时仿真功能,实时仿真周期可通过工具栏中的滑动条调节。在仿真时,会根据每根导线上电平的高低决定导线的不同颜色,方便用户观察。
    \par具体实现思路为为每一个端口记录0/1值,对应低电平与高电平。每一个仿真时间点内,遍历所有导线,将输出端的值赋给输入端,然后根据元件的calculate函数计算出每一个输出端口的新值,如此循环。
    \par仿真开始与结束时的具体操作:
    \par仿真开始:使用SimulationEditPolicy,隐藏所有端口,将所有端口值初始化为0,开始循环仿真;
    \par仿真结束:使用EditEditPolicy,保留端口值。

\subsubsection{波形编辑}
    \par波形编辑部分使用外部库dygraph实现。在点击编辑波形按钮时,会自动读取电路图内容,提取输入端口标签名,生成对应编辑界面。用户修改电路图或标签名时,波形编辑器会相应发生变化。在波形编辑时(过长输入端口名字会被隐藏),用户可以通过拖拽或生成周期波形设置输入端口的激励值,或是设置总仿真时间。如果对波形显示不满意,还可以调整缩放比例,以取得更佳观察效果。
    \par接口如下:
    \begin{table}[!h]
    \begin{tabular}{|l|l|}
    \hline
    名称 & 说明 \\
    \hline
    downV4(event, g, context) & 鼠标按下事件,调用startEdit() \\
    \hline
    moveV4(event, g, context) & 鼠标移动事件,调用moveEdit() \\
    \hline
    upV4(event, g, context) & 鼠标松开事件,调用endEdit() \\
    \hline
    startEdit (event, g, context) & 记录端口名、开始位置等数据,开始编辑波形 \\
    \hline
    moveEdit (event, g, context) & 计算距离等数据,编辑波形 \\
    \hline
    endEdit(event, g, context) & 根据之前的数据,生成波形 \\
    \hline
    onclickStart() & 设置仿真时间 \\
    \hline
    onclickReset() & 将波形图重置 \\
    \hline
    onclickPlot() & 生成周期波形 \\
    \hline
    reloadJili(projectId) & 将波形信息上传给服务器 \\
    \hline
    \end{tabular}
    \caption{波形编辑接口}
    \end{table}

\subsubsection{杂项}
    \par导线使用draw2d.connection实现,采用InteractiveManhattanConnectionRouter,允许用户拖拽折点和导线段来微调导线。
    \par电路图与服务器间的交流使用draw2d中的writer.marshal实现,以json格式传递。在仿真与提交前,程序会自动将每一条导线的source和destination属性调整为输出端与输入端,方便后端程序生成代码。
    \par用户观看样例工程时,提交相关功能会被禁止;用户编辑作业工程时,允许用户提交电路图进行标准激励的测试。


\subsection{VHDL代码编辑}
    \par文本编辑器以开源的ace-editor为基础,可为VHDL代码实现高亮功能,并添加了一些其他功能。

\subsubsection{代码补全}
    \par程序支持自动补全VHDL关键字,如begin,architecture等,出现对应补全框时按回车即可补全,有效提高了用户编辑速度。该功能实现文件在/TextEditor/src-nonconflict/mode-vhdl.js中。

\subsubsection{代码折叠}
    \par过高代码量会影响用户调试,而代码折叠功能可以让用户将不需要的代码段收起,只显示需要的部分,提高调试效率。在case,if,begin-end语句旁会出现小箭头,点击该箭头即可收起或放出对应代码段。
    \par使用的函数有:
    \begin{table}[!h]
    \begin{tabular}{|l|l|l|}
    \hline
    名称 & 内容 & 返回值 \\
    \hline
    getFoldWidget(session,foldStyle,row) & 从对应行号开始进行折叠 & 折叠信息类 \\
    \hline
    getBeginEndBlock (session, row, column, matchSequence) & 搜索可折叠代码段并返回位置 & 代码段的开始与结束位置 \\
    \hline
    \end{tabular}
    \caption{代码折叠接口}
    \end{table}

\subsubsection{多文件}
    \par文件树部分采用开源库ez-filetree,可以显示文件列表,并支持新建、删除、重命名、打开、保存等功能。多标签编辑文件采用了开源项目hhEditor。
    \par使用的函数有:
    \begin{table}[!h]
    \begin{tabular}{|l|l|}
    \hline
    名称 & 内容  \\
    \hline
    getFileTree() & 向后端请求文件列表 \\
    \hline
    activate(file) & 打开对应文件并新建标签显示 \\
    \hline
    \end{tabular}
    \caption{多文件接口}
    \end{table}

\subsubsection{快捷键}
    \par程序支持多种快捷键:Ctrl+s可以保存文件或将新建文件保存新内容;双击文件标签可以进行重命名;ctrl+shift+d可以删除相应文件。

\subsubsection{编译信息}
    \par编译完成后,用户可以点击左下角显示本次编译信息,在单击具体错误信息时,还会打开出错文件并跳转到对应行。


\subsection{项目信息显示}
    \par用户登录时,网站会从后端获取该用户的所有项目列表,根据项目complieStatus属性的不同,显示会有一些差异:
    \begin{table}[!h]
    \begin{tabular}{|l|l|l|}
    \hline
    值 & 意义 & 颜色  \\
    \hline
    0 & 仿真通过 & 绿色 \\
    \hline
    1 & 尚未提交 & 蓝色 \\
    \hline
    2 & 编译成功但仿真失败 & 黄色 \\
    \hline
    3 & 编译失败 & 红色 \\
    \hline
    \end{tabular}
    \caption{多文件接口}
    \end{table}
    \par用户的自定义项目、作业项目、仅供观看的样例项目会分三个不同区块显示。

\subsubsection{提交作业}
    \par在上传完电路图信息与激励波形后,用户可以进行提交,进行modelsim仿真。对于作业项目,用户还可以提交自己的电路图,以教师设定的标准激励为输入进行仿真。
    \par用户可以在首页或相应工程的编辑页面中查看每一次提交的信息。对于自定义激励的提交,可以查看到编译信息,点击对应条目还可以查看每一个输入输出端口的波形信息,此界面同样使用外部库dygraph实现。
    \par但对于标准激励的提交,只显示最终得分,不显示编译信息及波形,以防止学生获得标准激励信息。
    \par此外,用户可以下载自定义激励提交的所有文件(包括电路VHDL代码、激励代码等),方便调试。

\end{CJK}
\end{document}